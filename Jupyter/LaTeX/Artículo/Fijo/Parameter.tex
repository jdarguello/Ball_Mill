\usepackage[latin1]{inputenc}  % Para los caracteres tildados
%\usepackage[utf8]{inputenc}   % Para los caracteres tildados
%If your LaTeX compiler complains about utf8, try utf8x instead.
\usepackage[spanish]{babel}

\usepackage{amsmath,amsfonts}  % Fonts y macros de la American Math Society
\usepackage{amssymb, amsfonts, latexsym, cancel}
\usepackage{geometry}          % margenes de la manera facil
\usepackage{setspace}
\usepackage{hhline}
\usepackage{times}
\usepackage{color}
%\usepackage[sc]{mathpazo}
\usepackage{url}
\usepackage{fancyhdr}          % Fancy Header and Footer
\usepackage[ruled]{algorithm}       
\usepackage{algorithmic}
\usepackage{supertabular}
\usepackage{array}
\usepackage{abstract}
\usepackage{multicol}
\usepackage{titling}
\usepackage{titlesec}
\parindent=0pt

%\usepackage{setspace}          %para manejar single & double space
\usepackage{subfigure}

%~~~~~~~~~~~~~~~~~~~~~~~~~~~~~~~~~~~~~~~~~~~~~~~~~~~~~~~~~~~~~~~~~~~~~~~~~~~~  
%      PLEASE DO NOT MODIFY
%      POR FAVOR NO CAMBIAR

%~~~~~~~~~~~~~~~~~~~~~~~~~~~~~~~~~~~~~~~~~~~~~~~~~~~~~~~~~~~~~~~~~~~~~~~~~~~  
% Define the margins

\geometry{verbose, top=1.78cm,bottom=1.78cm, left=2cm, right=1.65cm}
\headheight=.76cm
\headsep=0pt

%~~~~~~~~~~~~~~~~~~~~~~~~~~~~~~~~~~~~~~~~~~~~~~~~~~~~~~~~~~~~~~~~~~~~~~~~~~~ % For use with PDFLaTeX
  \usepackage[pdftex]{graphicx} 
  \pdfcompresslevel=9 
  \usepackage[pdftex,bookmarksopen,colorlinks,linkcolor=blue,%
              citecolor=blue, urlcolor=blue]{hyperref} 

%~~~~~~~~~~~~~~~~~~~~~~~~~~~~~~~~~~~~~~~~~~~~~~~~~~~~~~~~~~~~~~~~~~~~~~~~~~
%   DEFINITIONS



\newcommand{\bx}{\mathbf{x}}
\newcommand{\bu}{\mathbf{u}}
\newcommand{\bd}{\mathbf{d}}
\newcommand{\by}{\mathbf{y}}
\newcommand{\bv}{\mathbf{v}}
\newcommand{\bbf}{\mathbf{f}}

\AtBeginDocument{\renewcommand{\abstractname}{}}

\def\Real{\mathbb{R}}%
\DeclareMathOperator{\volume}{volume}

%~~~~~~~~~~~~~~~~~~~~~~~~~~~~~~~~~~~~~~~~~~~~~~~~~~~~~~~~~~~~~~~~~~~~~~~~~~
%     SECTION FORMAT
\mathindent=10pt
\makeatletter
\abovecaptionskip=4pt
\renewcommand{\section}{\@startsection%
   {section}%
   {1}%
   {\z@}%
   {-\baselineskip}%
   {0.2\baselineskip}%
   {\bfseries\boldmath\MakeUppercase}}
\def\subsection{\@startsection%
   {subsection}%
   {2}%
   {\z@}%
   {-\baselineskip}%
   {0.2\baselineskip}%
   {\noindent\normalsize\boldmath\emph}}
\def\subsubsection{\@startsection{subsubsection}
   {3}
   {\z@}%
   {-13dd plus-4pt minus-4pt}
   {-5.5pt}
   {\normalsize\emph}}

\renewcommand\paragraph{\@startsection{paragraph}{3}{\z@}%
            {-2.5ex\@plus -1ex \@minus -.25ex}%
            {1.25ex \@plus .25ex}%
            {\normalfont\normalsize\bfseries}}

\makeatother
\setcounter{secnumdepth}{3} % how many sectioning levels to assign numbers to
\setcounter{tocdepth}{3}    % how many sectioning levels to show in ToC