\section{Introduction} 

\subsection{Background and Motivation}

In recent years, there have been reported several studies where automation has been used in engineering design. For the production of nuclear energy, it can be found different studies, such as: the design of reactor cores$^{[1]}$, from where Hyun Kim \textit{et al} uses an artificial neural network to provide, and evaluate, new types of geometries; Nomen \textit{et al} developed a numerical methodology to create the design of IFMIF LIPAc beam dump shielding$^{[2]}$, which provides an acelerator-based, D-Li neutron source to produce energy of high intensity by radiation; Kim J. \textit{et al} created an autonomous operation system$^{[3]}$ with artificial intelligence which perform the control functions needed for the emergency operation of a nuclear power plant.


    In civil engineering, there is a specialized journal known as "Automation in Construction", which has published different studies related to automatic structural and seismic designs. In mechanical design, it can be found articles related with automation in different industries, like automotive, fluid transport and robotics, to name a few. For example: Ouyang T. \textit{et al} developed a dynamic modelling of a clutch actuator for heavy duty transmission$^{[4]}$ adopting an artificial bee colony (ABC) algorithm to optimize structural parameters; X. Telleria \textit{et al} presented a methodology to automate the design, and numerical validation, of valves$^{[5]}$; M. Honarpardaz \textit{et al} introduced the Generic Automated Finger Design$^{[6, 7]}$ (GAFD) method for design automation of customized fingers of industrial grippers for robots. 

   Most of these works are intended to study a particular stage of the design, either modelling, numerical method or simulation. The motivation of doing this work was to propose a design methodology, which groups all of this efforts, focusing in the required information for the manufacturing of the mill.
   
\subsection{Future Updates}

Things can always be done better or differently. The main idea of this work was to prove a design methodology, but the possibility to do things "simpler" from the user perspective is still there. To achieve this, an artificial intelligence algorithm must be implemented. 

    Another future update should let advanced users to implement experimental grinding equations to predict time consuming.